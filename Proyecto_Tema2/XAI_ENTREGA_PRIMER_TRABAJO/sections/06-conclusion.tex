\section{Conclusión}

El estudio comparó tres modelos explicables para la predicción de reincidencia penal: un árbol de decisión, una regresión logística y un modelo aditivo (Explainable Boosting Machine, EBM). 
Los resultados muestran que todos alcanzan un rendimiento competitivo, con \textbf{accuracies medias de 0.719, 0.702 y 0.723} respectivamente, manteniendo una buena estabilidad entre entrenamiento y validación. 
El uso de un pipeline reproducible garantizó coherencia en el preprocesamiento y permitió comparar de forma justa la capacidad predictiva y explicativa de cada enfoque.

A nivel de \textbf{explicabilidad global}, los tres modelos coincidieron en resaltar las mismas variables clave: la situación laboral, el número de antecedentes y la edad. 
El árbol ofreció una lectura jerárquica y visual a través de reglas de decisión claras, la regresión logística tradujo las relaciones en coeficientes de interpretación directa y el EBM amplió esta visión mediante curvas de efecto que capturan relaciones no lineales con notable claridad. 
Esta convergencia entre modelos refuerza la fiabilidad de los patrones identificados y demuestra la coherencia del conjunto de datos.

En cuanto a la \textbf{explicabilidad local}, cada modelo aportó una perspectiva complementaria: el árbol permite seguir paso a paso la ruta de decisión de un individuo, la regresión logística descompone las predicciones en contribuciones lineales de cada variable, y el EBM combina ambas visiones, mostrando de forma visual cómo cada factor incrementa o reduce el riesgo individual de reincidencia. 
Esta capacidad de justificar cada decisión con detalle convierte al EBM en la opción más completa desde el punto de vista explicativo.

En conjunto, los resultados confirman que la transparencia y la precisión pueden coexistir. 
El árbol destaca por su claridad interpretativa, la regresión logística por su consistencia estadística y el EBM por integrar ambas virtudes, alcanzando el mejor equilibrio entre rendimiento, estabilidad y capacidad explicativa. 
Por todo ello, el EBM se posiciona como el modelo más sólido del estudio, al lograr unir de forma efectiva la \textbf{precisión predictiva y la explicabilidad}, demostrando que la inteligencia artificial puede ser al mismo tiempo rigurosa y comprensible.

